\documentclass{article}
\usepackage{amsmath}

\begin{document}

The variables used in the Active Inference Thermostat agent:

\begin{enumerate}
	\item $temperature$: the current temperature of the system being controlled by the Active Inference Thermostat agent.
	\item $T$: The target temperature, which is the temperature that the thermostat is trying to maintain.
	\item $P$: The prior belief about the temperature, which is used to update the posterior belief based on new sensory information.
	\item $E$: The energy or prediction error, which is the difference between the predicted temperature and the actual temperature.
	\item $H$: The entropy or uncertainty about the prediction error, which represents the degree of confidence in the prediction error.
	\item $F$: The free energy, which is the difference between the entropy and the energy. It is a measure of the surprise or unexpectedness of the sensory input.
	\item $k$: The control precision, which is the inverse of the expected variance of the control signal. It determines the strength of the control action.
	\item $\gamma$: The precision parameter of the noise model, which represents the precision of the sensory input.
	\item $\phi$: The free energy contribution from the noise model, which represents the cost of acquiring sensory information.
	\item $\mu$: The mean of the posterior distribution, which represents the best estimate of the temperature.
	\item $\sigma$: The standard deviation of the posterior distribution, which represents the uncertainty or precision of the estimate.
	\item $\alpha$: The learning rate, which determines the speed at which the prior parameters are updated based on new sensory information.
\end{enumerate}

Here are the key equations that describe the operation of the Active Inference Thermostat agent.

1. Generative model:
The generative model takes in the current temperature ($temperature$) and the parameters of the prior distribution ($T$, $P$, $\mu$, and $\sigma$) and outputs the precision-weighted prediction error ($pe_{pw}$). Mathematically, we can express this as:

\begin{equation}
pe_{pw} = \frac{1}{\sigma^2}(T\mu + P - temperature)
\end{equation}

2. Free energy:
The free energy is calculated based on the precision-weighted prediction error ($pe_{pw}$), the parameters of the noise model ($\phi$ and $\gamma$), and the control parameter ($k$). Mathematically, we can express this as:

\begin{equation}
E = \frac{1}{2}pe_{pw}^2
\end{equation}

\begin{equation}
H = \frac{\gamma}{2}\ln(2\pi) + \frac{\gamma}{2}\ln(k^2) + \frac{\phi}{2\gamma}
\end{equation}

\begin{equation}
F = E + H
\end{equation}

3. Active inference update:
The active inference update takes in the generative model, the current free energy ($F$), the prior parameters ($T$, $P$, $\mu$, and $\sigma$), the control parameter ($k$), the current temperature ($temperature$), and the learning rate ($\alpha$), and outputs updated values for the prior parameters and the free energy. Mathematically, we can express this as:

\begin{equation}
\frac{\partial F}{\partial T} = -\frac{pe_{pw}}{k^2}\mu
\end{equation}

\begin{equation}
\frac{\partial F}{\partial P} = -\frac{pe_{pw}}{k^2}
\end{equation}

\begin{equation}
\frac{\partial F}{\partial \mu} = \frac{pe_{pw}}{k^2}T
\end{equation}

\begin{equation}
\frac{\partial F}{\partial \sigma} = -\frac{pe_{pw}}{k^2}\sigma + \frac{\gamma}{\sigma}
\end{equation}

\begin{equation}
T_{new} = T - \alpha\frac{\partial F}{\partial T}
\end{equation}

\begin{equation}
P_{new} = P - \alpha\frac{\partial F}{\partial P}
\end{equation}

\begin{equation}
\mu_{new} = \mu - \alpha\frac{\partial F}{\partial \mu}
\end{equation}

\begin{equation}
\sigma_{new} = \sigma - \alpha\frac{\partial F}{\partial \sigma}
\end{equation}

\begin{equation}
pe_{pw,new} = \frac{1}{\sigma_{new}^2}(T_{new}\mu_{new} + P_{new} - temperature)
\end{equation}

\begin{equation}
F_{new} = E_{new} + H
\end{equation}

4. Control action:
The control action is based on the current free energy and the set point temperature ($setpoint$). If the free energy is greater than a threshold value ($F_{threshold}$), the control action decreases the temperature by a fixed amount ($\Delta$). Otherwise, the control action increases the temperature by a fixed amount ($\Delta$). Mathematically, we can express this as:

$$\Delta T = \begin{cases} -\Delta, & F > F_{threshold} \\ \Delta, & F \leq F_{threshold} \end{cases}$$

$$T_{new} = T + \Delta T$$

Once the control action has been calculated, the updated value of the temperature ($T_{new}$) is passed back to the generative model, which in turn generates a new prediction error ($pe_{pw,new}$). This prediction error is then used in the active inference update step to compute updated values for the prior parameters and the free energy, and the process repeats.

\end{document}
